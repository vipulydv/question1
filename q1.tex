\documentclass{article}
\usepackage{algorithm}
\usepackage{algpseudocode}
\usepackage{ragged2e}
\usepackage{graphicx}
\usepackage[normalem]{ulem}
\usepackage{enumitem}
\usepackage{hyperref}
\usepackage{amsmath}
\usepackage{geometry}
 \geometry{
 a4paper,
 total={170mm,257mm},
 left=20mm,
 right=20mm,
 top=5mm,
 bottom=10mm
 }

\title{ASSIGNMENT 7}
\author{Vipul Yadav}
\date{6 March 2017}

\begin{document}

\maketitle

\section{INTRODUCTION}
\label{sec1}

In this assignment we look at multiplying two digit numbers when they are of the form:
\begin{itemize}[leftmargin=\dimexpr\parindent+0\labelwidth\relax]
     \item The first digits are same, and the last ones add up to 10.
     \item The first digits add up to 10, and the last ones are same.
\end{itemize}
To know more about this method refer \cite{vedic}, \cite{florentinwebsite} and \cite{commentry}

\section{The first digits are same, and the last ones add up to 10}
\label{sec2}
The following example follows the above condition\footnote{The first digits are same, and the last ones add up to 10}.

\subsection{How to compute the result mentally?}
\label{subsec1}

Here, we see that the first digits of given numbers are same, and the last digits add up to 10.So,to get the answer numbers we follow these steps:
\begin{itemize}[leftmargin=\dimexpr\parindent+0\labelwidth\relax]
     \item Step 1: Multiply the first digits which are same with its successor (i.e. 6*7 in the given example) and let the answer be a1.
     \item Step 2: Multiply the right part of the numbers which are not same (i.e. 6*4 in the given example) and let the answer be a2.
     \item Step 3: The final answer is a1 followed by a2 (i.e. 4224)
\end{itemize}


\subsection{Pseudocode for this algorithm.}
\label{subsec2}
\begin{algorithm}
   \caption{Vedic Multiplication}
    \begin{algorithmic}[h]
      \Function{Mult}{$X,Y$}

        \State $i \gets units \thinspace place \thinspace of \thinspace X $
        \State $j \gets units \thinspace place \thinspace of \thinspace Y $
        \State $k \gets tens \thinspace place \thinspace of \thinspace X $
        \State $l \gets i*j$
        \State $m \gets k*(k+1)$
        \newline
        \Return $(100*m)+l$
       \EndFunction
\end{algorithmic}
\end{algorithm}

\subsection{Why is this method easier and faster?}
\label{subsec3}
This method is much more faster than normal two digit multiplication as it involves only 2 multiplication steps whereas in normal multiplication we have to add 4 times and multiply 4 times.
\newline

\begin{center}
{\begin{tabular}{| p{3cm} | p{4cm} | p{4cm} |}
\hline
Method & 1 digit Multiplications & Additions \\
\hline
Vedic Math  & \textit{2}  & 0  \\

\hline

Traditional Algebra          &  4         & 4   \\

\hline
\end{tabular}
}
\end{center}


\subsection{Why does this method work?}
\label{subsec4}
Suppose a pair of two digit numbers X and Y have the same first digit $x$ and their last digits sum to 10. We can write X and Y as

\begin{align}
    Y&=10x+(10-y) \\
    X*Y&=(10x+y)*(10x+(10-y)) \\
    X*Y&=100x^2+10x(10-y)+10xy+(10-y)y \\
    X*Y&=100x^2+100x+10y-y^2 \\
    X*Y&=100x(x+1)+y(10-y)
\end{align}
The first part of this expression, $100x(x+1)$ is the first two digits of the product. The factor of 100 makes the product $x(x+1)$ occur two places before the decimal point. The last part of this expression, $y(10-y)$,is the product of the last two digits of X and Y. Putting both pieces together gives us the product of X and Y.

\subsection{Generalization of this method}
\label{subsec5}
The idea for generalization is inspired from 3 digit number multiplication as given on \href{http://www.vedantatree.com/2012/05/vedic-math-multiplication-of-numbers.html}{this} website.
Suppose a pair of two digit numbers X and Y have $n$ digits each and let the first $m$ digits of these numbers be same and the sum of next $n-m$ digits be $10^{n-m}$ then to get the product of these numbers, the following procedure is to be used which is similar to \ref{subsec1}.
\begin{itemize}[leftmargin=\dimexpr\parindent+0\labelwidth\relax]
     \item Step 1: Multiply the left part of the number(i.e. the part which is common to both the numbers) with its successor and let the answer be a1.
     \item Step 2: Multiply the right part of the numbers which are not same  and let the answer be a2.
     \item Step 3: The final answer is a1 followed by a2 (where a2 takes $n-m+1$ places before decimal).
\end{itemize}

\section{The first digits add up to 10, and the last ones are same.}
\label{sec3}
The following example follows the above condition\footnote{The first digits add up to 10, and the last ones are same.}

\subsection{How to compute the result mentally?}
\label{subsec6}
Here, we see that the first digits of given numbers are same, and the last digits add up to 10.So,to get the answer numbers we follow these steps:
\begin{itemize}[leftmargin=\dimexpr\parindent+0\labelwidth\relax]
     \item Step 1: Multiply the first digits which are not same and add the number which is same to these numbers to the result (i.e. (6*4)+6 in the given example) and let the answer be a1.
     \item Step 2: Multiply the right part of the numbers which is same (i.e. 6*6 in the given example) and let the answer be a2.
     \item Step 3: The final answer is a1 followed by a2 (i.e. 3036)
\end{itemize}

\subsection{Pseudocode for this algorithm.}
\label{subsec7}
\begin{algorithm}
   \caption{Vedic Multiplication}
    \begin{algorithmic}[h]
      \Function{Mult}{$X,Y$}

        \State $i \gets units \thinspace place \thinspace of \thinspace X $
        \State $j \gets tens \thinspace place \thinspace of \thinspace X $
        \State $k \gets tens \thinspace place \thinspace of \thinspace Y $
        \State $l \gets (j*k)+i$
        \State $m \gets i^2$
        \newline
        \Return $100l+m$
       \EndFunction
\end{algorithmic}
\end{algorithm}



\end{document}
